\documentclass[a4paper,12pt]{scrartcl}

\usepackage[margin=25mm,bottom=30mm]{geometry}
\usepackage[utf8]{inputenc}
\usepackage[brazil]{babel}
\usepackage[pdftex]{hyperref}
\usepackage[normalem]{ulem}
\usepackage{xfrac}
\usepackage{minted}
\usepackage{csvsimple}
\usepackage{siunitx}


\title{Relatório do Trabalho de Ordenação e Estatísticas de Ordem}
\subtitle{Técnicas de Programação Avançada --- Ifes --- Campus Serra}
\author{\uline{Aluno: INSIRA O NOME DO ALUNO}%
  \and Prof. Jefferson O. Andrade}
\date{29 de setembro de 2018}


% Define o texto em português que será impresso pelo ambiente `listing', e pelo
% comando `\listoflistings'. Infelizmente o pacote babel não inclui estas
% traduções.
\renewcommand{\listingscaption}{Código Fonte}
\renewcommand{\listoflistingscaption}{Lista de Códigos Fonte}

% Comando necessário para que o `autoref' funcione corretamente para listagens
% de códigos fonte, i.e., para o ambiente `listing'.
\providecommand*{\listingautorefname}{Código Fonte}


%%% ============================================================================
%%% ============================================================================
\begin{document}

\maketitle

\tableofcontents

\listoflistings

\listoffigures


\section{Introdução}
\label{sec:introducao}

Este documento tem o propósito duplo de servir de modelo para confecção do
relatório que deve ser entregue como parte do Trabalho de Ordenação e
Estatísticas de Ordem da disciplina de Técnicas de Programação Avançada, e
também de descrever dois algoritmos de ordenação não discutidos em sala de aula.


\subsection{Ambiente de Desenvolvimento}
\label{sec:amb-desenv}

Para o desenvolvimento do trabalho foi utilizada a linguagem Scala, e para
automação de algumas das atividades de preparação de dados foram utilizados
scripts do interpretador de comandos (shell) Bash.

Para edição do código fonte foi utilizado o editor de textos
\href{https://www.gnu.org/software/emacs/}{GNU Emacs}, com a distribuição
\href{http://spacemacs.org/}{Spacemacs}. Para suporte ao desenvolvimento de
código em Scala foi utilizado o \emph{plug-in}
\href{https://ensime.github.io/}{ENSIME}, do Emacs, que acrescenta
características típicas de um ambiente integrado de desenvolvimento.


\section{Geração de Dados para Testes}
\label{sec:geracao-de-dados}

Originalmente, pretendia-se utilizar um sistema online de geração de dos
aleatórios para testes. Entretanto, todos os sistemas online gratuitos
encontrados tinham severas limitações de número de registros que poderiam ser
criados. Tipicamente limitando este número em 500 registros ou menos. Como seria
necessária a geração de arquivos com mais de 1 milhão de registros para os
testes dos algoritmos de ordenação, a opção de utilizar um sistemas online ficou
inviabilizada.

Deste modo, foi desenvolvido um programa específico para geração dos dados para
teste dos algoritmos de ordenação. Este programa foi desenvolvido na linguagem
Scala, e é um programa relativamente simples.


\subsection{Programa para Geração de Dados}

O programa desenvolvido foi chamado de \textsf{Mokata}, que é um amálgama entre
as palavras em ingês \emph{mock} e \emph{data}. A versão atual do
\textsf{Mokata} (versão 0.1.0) foi desenvolvida especificamente para gerar os
dados para este trabalho, o que levou a um programa pouco flexível e com
diversas decisões tomadas arbitrariamente e expressa de modo fixo em código,
i.e., \emph{hard-coded}. Espera-se que o \textsf{Mokata} evolua para,
eventualmente, poder ser usado para geração de dados aleatórios de testes para
uma grande variedade de programas, podendo ser configurado para gerar uma
quantidade variável de campos, e uma grande variedade de tipos diferentes de
campos de dados, de modo consistente. Por exemplo, mantendo consistência entre
nome e gênero, ou entre nome e e-mail.

O programa \textsf{Mokata} recebe dois argumentos obrigatórios, o número de
registros que devem ser gerado e o nome do arquivo de saída onde os registros
devem ser gravados, e um argumento opcional, a semente aleatória que será usada
para a geração dos dados. Se a semente não for especificada, o valor padrão
$12345$ será utilizado.

\begin{listing}[h]
\begin{minted}[autogobble,breaklines,linenos,fontsize=\small]{scala}
  def genMockData(config: Config) {
    val writer = CSVWriter.open(config.outputFile)
    val header = List("email", "gender", "uid", "birthdate", "height", "weight")
    writer.writeRow(header)

    var uidSet: TreeSet[Long] = TreeSet.empty
    for (i <- 1 to config.recordCount) {
      val row = genMockRow(uidSet)
      writer.writeRow(row)
    }
    writer.close()
  }
\end{minted}
\caption[Função de geração do arquivo de dados.]{Função com o loop principal
  que gera o arquivo de dados do programa \textsf{Mokata}.}
\label{lst:genmockdata}
\end{listing}

O~\autoref{lst:genmockdata} mostra a função contendo o loop principal do
programa \textsf{Mokata} que gera as linhas de dados individualmente e as grava
no arquivo de saída. A função \mintinline{scala}{genMockData} recebe os
argumentos de configuração do programa através do parâmetro \texttt{config}.

\begin{listing}[h]
\begin{minted}[autogobble,breaklines,linenos,fontsize=\small]{scala}
  def genMockRow(uidSet: TreeSet[Long]): List[String] = {
    val email = genMockEmail
    val gender = genMockGender
    val uid = genMockUserId(uidSet)
    val bdate = genMockBirthdate.toString
    val height = genMockHeight(175, 15)
    val weight = genMockWeight(height)
    val row = List(email, gender, uid, bdate, height.toString, round(weight).toString)
  }
\end{minted}
\caption{Função que gera cada linha individual de dados.}
\label{lst:genmockrow}
\end{listing}

O~\autoref{lst:genmockrow} mostra a função que gera as linhas de dados
individuais. Note que existe uma \emph{dependência funcional estatística} entre
altura e peso. A função \mintinline{scala}{genMockWeight} recebe a altura como
parâmetro, juntamente com um indicador de desvio padrão para o peso. O peso é
calculado aleatoriamente, porém com base em uma distribuição probabilística
gaussiana com centro no \emph{Índice de Massa Corporal} (BMI) 25 e desvio padrão
de acordo com o indicado como parâmetro. Outras dependências deverão ser criadas
em futuras versões do programa, à medida que novos tipos de campos forem
acrescentados à ele.


\subsection{Bibliotecas Externas e Build}

Para a análise e processamento dos argumentos de linha de comando foi utilizado
a biblioteca \href{https://github.com/scopt/scopt}{scopt}. Esta biblioteca foi
escolhida fundamentalmente por sua simplicidade de uso. Embora haja bibliotecas
mais sofisticadas, estas exigiam maior tempo de estudo para o aprendizado, e os
recursos adicionais oferecidos não seriam necessários para o que tínhamos em
mente.

De modo semelhante, a escolha da biblioteca utilizada para a gravação dos
arquivos CSV foi feita baseada na simplicidade de uso. A biblioteca escolhida
foi a \href{https://github.com/tototoshi/scala-csv}{scala-csv}. Esta biblioteca
define um \emph{reader} e um \emph{writer} para arquivos CSV que retornam listas
de \emph{strings} e gravam listas de \emph{strings}, respectivamente. Embora a
biblioteca não ofereça muito, é o suficiente para as necessidades do
\textsf{Mokata} e o tempo de aprendizado é extremamente curto -- a página da
biblioteca no GitHut apresenta um conjunto de exemplos que permite saber o
necessário para utilizá-la em menos de 5 minutos.

\begin{listing}[h]
\begin{minted}[autogobble,breaklines,linenos,fontsize=\small]{scala}
lazy val root = (project in file("."))
  .settings(
    name := "Mokata",
    version := "0.1.0",
    scalaVersion := "2.12.6",
    libraryDependencies += "com.github.scopt" %% "scopt" % "3.7.0",
    libraryDependencies += "com.github.tototoshi" %% "scala-csv" % "1.3.5"
  )
\end{minted}
\caption{Arquivo de configuração do projeto \textsf{Mokata} no sbt.}
\label{lst:mokata-build-sbt}
\end{listing}

Para o processo de \emph{build} foi utilizada a ferramenta
\href{https://www.scala-sbt.org/}{sbt}. O~\autoref{lst:mokata-build-sbt} mostra
o arquivo de configuração da ferramenta sbt para o projeto \emph{Mokata},
permitindo ver a configuração para inclusão das bibliotecas scopt e scala-csv.

O sbt também é necessário para integração do \emph{plug-in} ENSIME ao editor
Emacs, visto que o arquivo de configuração do ENSIME é gerado pelo comando
\emph{ensimeConfig} acrescentado ao sbt através de um \emph{plug-in} do sbt
desenvolvido pela equipe do ENSIME.

O processo padrão de ``\emph{packaging}'' do sbt gera um arquivo JAR que pode
ser executado no \emph{Java Virtual Machine} (JVM), porém com dependência. Para
que esse arquivo JAR seja executado pela JVM é necessário indicar no
\emph{classpath} o arquivo JAR com a biblioteca padrão da linguagem Scala, bem
como os arquivos JAR com as bibliotecas de terceiros (neste caso a scopt e a
scala-csv). Esse processo é inconveniente. Para reduzir essa inconveniência, foi
usado o \emph{plug-in} \href{https://github.com/sbt/sbt-assembly}{sbt-assembly}.
Este \emph{plug-in} cria arquivos JAR que já contém todas as dependências do
projeto, eliminando a necessidade do usuário ter que baixar essas dependências e
especificá-las como parâmetros, ou na variável de ambiente CLASSPATH.


\subsection{Geração dos Arquivos de Dados}

Para automatizar o processo de geração dos arquivos de dados foi gerado um
script do interpretador de comandos Bash. O código fonte deste script é mostrado
como apêndice na~\autoref{lst:mkdata}.

O script da~\autoref{lst:mkdata}, gravado no arquivo \texttt{mkdata.sh}, define
três listas: \texttt{PRIMES}, \texttt{SIZES}, e \texttt{NAMES}. A lista
\texttt{PRIMES} contém números primos de 5 dígitos que serão usados como semente
de geração de números randômicos pelo programa \textsf{Mokata}. A lista foi
obtida no site
\href{https://primes.utm.edu/curios/index.php?start=5&stop=5}{Prime Curious}. A
lista \texttt{SIZES} contém o tamanho (número de registro) dos arquivos de serão
gerados, e a lista \texttt{NAMES} contém uma lista de identificadores que serão
utilizados para compor o nome dos arquivos que serão gerados. É necessário que a
lista \texttt{PRIMES} e a lista \texttt{NAMES} tenham pelo menos a mesma
quantidade de elementos da lista \texttt{SIZES}. Caso contrário, ocorrerá um
erro na execução do script.



%%% ============================================================================
\section{Implementação do Trabalho}

Como dito anteriormente, o trabalho foi implementado na linguagem Scala,
utilizando o sistema de \emph{build} sbt, e Emacs/ENSIME como ambiente de
desenvolvimento. O projeto consiste de apenas três arquivos:

\begin{itemize}
\item \texttt{MySort.scala} -- Contém o código para o programa principal.
\item \texttt{Sorting.scala} -- Contém a implementação dos algoritmos de
  ordenação.
\item \texttt{Utils.scala} -- Contém a implementação da função que conta o
  número de linhas de um arquivo.
\end{itemize}


\subsection{Programa Principal}

O programa principal é responsável por processar os argumentos de linha de
comando, ler os dados do arquivo CSV de entrada, executar o algoritmo de
ordenação, coletar o tempo de execução do algoritmo de ordenação, e gravar os
dados ordenados no arquivo CSV de saída. 

\begin{listing}[h]
\begin{minted}[autogobble,breaklines,linenos,fontsize=\small]{scala}
  def run(config: Config): Unit = {
    validateConfig(config);

    val persons = loadInputFile(config.inputFile)

    val sortInitTime = System.nanoTime()
    config.algorithmId match {
      case "shell" => Sorting.shellsort(persons)
      case "comb" => Sorting.combsort(persons)
      case _ => throw new RuntimeException(
        s"Error! Invalid algorithm id: ${config.algorithmId}")
    }
    val sortFinishTime = System.nanoTime()

    writeOutputFile(config.outputFile, persons)
    reportTime(config.algorithmId, persons.size, sortFinishTime - sortInitTime)
  }
\end{minted}
\caption{Função principal do programa \textsf{MySort}.}
\label{lst:mysort-run}
\end{listing}

O~\autoref{lst:mysort-run} mostra a função principal do programa
\textsf{MySort}. Optamos por seguir a arquitetura proposta no Algoritmo 2 do
enunciado do trabalho, ou seja, ter um único programa principal que recebe como
parâmetro qual algoritmo de ordenação deve ser aplicado. É possível ver nas
linhas de 7 a 12 do~\autoref{lst:mysort-run} que o campo \texttt{algorithmId} da
configuração do programa é usado para selecionar o algoritmo.

\begin{listing}[h]
\begin{minted}[autogobble,breaklines,linenos,fontsize=\small]{scala}
  def loadInputFile(inputFile: File): Array[Person] = {
    val numLines = Utils.countLines(inputFile)
    val persons = new Array[Person](numLines - 1)
    val reader = CSVReader.open(inputFile)
    reader.readNext()
    var i = 0
    reader foreach {fields => persons(i) = Person.fromList(fields); i += 1}
    reader.close()
    persons
  }
\end{minted}
\caption{Função que faz a leitura dos registros do arquivo de entrada.}
\label{lst:mysort-load-input}
\end{listing}

A função \mintinline{scala}{loadInputFile}, merece um comentário à parte. Uma
forma relativamente simples de fazer a leitura do arquivo de entrada seria
construir uma lista simplesmente encadeada à medida que fôssemos lendo os
registros e, ao final, retornar esta lista. Entretanto, precisamos de um vetor
para passar aos algoritmos de ordenação, então as listas teriam que ser
convertidas em vetores. Eta foi nossa primeira abordagem. Porém esta abordagem
estava causando um erro de exaustão de memória com o arquivo de 5 milhões de
registros. Em virtude disso, foi necessário mudar de estratégia, e passamos a
utilizar uma função que conta o número de linhas do arquivo, então alocamos o
vetor já com seu tamanho total e em seguida passamos a ler os registros e
armazená-los já diretamente no vetor. O~\autoref{lst:mysort-load-input}
apresenta a função que faz a leitura dos arquivos de entrada.


\subsection{Estruturas de Dados}
\label{sec:estruturas-de-dados}

Para modelar os dados que o programa deve manipular foi definida uma estrutura
de dados, ou uma \emph{case class} na terminologia de Scala, chamada
\mintinline{scala}{Person}. Esta estrutura de dados é composta por seis campos,
correspondentes as colunas dos arquivos CSV de entrada.

\begin{listing}[h]
\begin{minted}[autogobble,breaklines,linenos,fontsize=\small]{scala}
  case class Person(uid: String, email: String, gender: Char,
                    birthdate: LocalDate, height: Int, weight: Int)
  {
    def toList: List[String] = {
      List(email,
           gender.toString,
           uid,
           birthdate.toString,
           height.toString,
           weight.toString)
    }
  }
\end{minted}
\caption{Definição da estrutura de dados \texttt{Person}.}
\label{lst:person}
\end{listing}

O~\autoref{lst:person} apresenta a definição do tipo \mintinline{scala}{Person}.
Note que também foi definida para valores este tipo uma operação (método)
chamada \mintinline{scala}{toList} que converte o objeto uma lista de strings
correspondente. Esta operação será utilizada para gravação dos dados no arquivo
de saída.

\begin{listing}[h]
\begin{minted}[autogobble,breaklines,linenos,fontsize=\small]{scala}
  object Person {
    def fromList(fields: Seq[String]): Person = {
      val email = fields(0)
      val gender = fields(1).charAt(0)
      val uid = fields(2)
      val birthdate = LocalDate.parse(fields(3))
      val height = fields(4).toInt
      val weight = fields(5).toInt
      Person(uid, email, gender, birthdate, height, weight)
    }
  }
\end{minted}
\caption{Objeto companheiro da classe \texttt{Person}, que define a operação
\texttt{fromList}.}
\label{lst:person-obj}
\end{listing}

O~\autoref{lst:person-obj} apresenta a definição do \emph{objeto companheiro}
(\emph{companion object}) da classe \mintinline{scala}{Person}. Esse objeto
define uma função chamada \mintinline{scala}{fromList} que realiza a operação
inversa da função \mintinline{scala}{toList}, i.e., recebe uma lista de strings
e constrói um objeto do tipo \mintinline{scala}{Person} à partir dos valores da
lista.

\begin{listing}[h]
\begin{minted}[autogobble,breaklines,linenos,fontsize=\small]{scala}
  implicit val ordPerson: Ordering[Person] = new Ordering[Person]() {
    def compare(x: Person, y: Person): Int = x.uid.compare(y.uid)
  }
\end{minted}
\caption[Declaração implícita de \texttt{Ordering[Person]}.]{Definição do valor
implícito do tipo \mintinline{scala}{Ordering[Person]} que será usado como forma
padrão de comparação entre dois valores do tipo \mintinline{scala}{Person}.}
\label{lst:ordering-person}
\end{listing}

O~\autoref{lst:ordering-person} apresenta a definição de um valor implícito do
tipo \mintinline{scala}{Ordering[Person]}. Esta interface define uma função de
comparação que será usada como padrão toda vez que uma operação qualquer definir
um parâmetro deste tipo como implícito. Os algoritmos de ordenação implementados
neste trabalho utilizaram esta estratégia, i.e., definiram parâmetros implícitos
do tipo \mintinline{scala}{Ordering[T]}, de modo que ao chamar as funções de
ordenação não é necessário passar explicitamente o objeto que faz a comparação.


\subsection{Algoritmos de Ordenação}
\label{sec:algor-ordenacao}

O Trabalho de Ordenação e Estatísticas de Ordem pede a implementação de cinco
algoritmos de ordenação diferentes. A saber: ordenação por seleção
(\emph{selection sort}), ordenação por inserção (\emph{insertion sort}),
\emph{merge sort}, \emph{quicksort}, e \emph{heapsort}.

Neste exemplo de relatório, entretanto, foram implementados dois outros
algoritmos que não estão na lista pedida. Para não estragar a diversão dos
alunos durante a implementação.

Os algoritmos implementados são o \emph{Shell sort} e o \emph{Comb sort}, ambos
podem ser considerados generalizações de algoritmos já vistos, e apresentam boas
performances.


\subsubsection{Algoritmo Shell Sort}
\label{sec:algoritmo-shell-sort}

O \emph{Shell sort} foi desenvolvido por Donald Shell. Este algoritmo utiliza a
ideia de enxergar a sequência a ser ordenada como sendo composta por várias
subsequencias intercaladas com um certo intervalo (\emph{gap}) entre elementos
de cada sequência. Por exemplo, suponhamos um vetor de 10 posições (0 a 9), e um
intervalo de 5 isso geraria 5 subsequencias de 2 elementos: 0 e 5; 1 e 6; 2 e 7;
3 e 8; 4 e 9. Cada subsequência é ordenada por \emph{insertion sort}. Em seguida
diminui-se o \emph{gap} e repete-se o processo. Qualquer sequência de
\emph{gaps} que termine com um \emph{gap} de tamanho 1 garante a ordenação
(porque com o \emph{gap} de tamanho 1 o \emph{shell sort} se torna idêntico ao
\emph{insertion sort}).

A grande melhoria do \emph{shell sort} em relação ao \emph{insert sort} se dá
nas primeiras iterações, com valores de \emph{gap} maiores, que faz com que
valores pequenos que estejam no final do vetor sejam rapidamente movidos para as
posições iniciais.

A complexidade do \emph{shell sort} depende muito dos tamanhos dos \emph{gaps}
escolhidos e pode variar, no pior caso, desde $\Theta\left(n^2\right)$ até
$\Theta\left(n^{\sfrac{4}{3}}\right)$. A página da Wikipédia sobre
\href{https://en.wikipedia.org/wiki/Shellsort}{Shellsort} possui um apanhado
sobre a complexidade do \emph{shell sort} para várias sequências de \emph{gaps}
diferentes que já foram propostas na literatura.

\begin{listing}
\begin{minted}[autogobble,breaklines,linenos,fontsize=\small]{scala}
  def gterm(k: Int): Int = pow(4, k).toInt + 3 * pow(2, k-1).toInt + 1

  val gapStream: Stream[Int] = 1 #:: (Stream.from(1) map {k => gterm(k)})

  def shellsort[T](a: Array[T])(implicit ord: Ordering[T]): Array[T] = {
    val gaps = (gapStream takeWhile {g => g < a.size}).toSeq.reverse
    for (gap <- gaps) {
      for (i <- gap until a.size) {
        val temp = a(i)
        var j = i
        while (j >= gap && ord.compare(a(j-gap), temp) > 0) {
          a(j) = a(j - gap)
          j -= gap
        }
        a(j) = temp
      }
    }
    a
  }
\end{minted}
\caption{Implementação do algoritmo Shell sort.}
\label{lst:shellsort}
\end{listing}

Nossa implementação utiliza a sequência de \emph{gaps} proposta por Robert
Sedgewick, onde cada termo de ordem $k$ da sequência é definido pela expressão
$g(k) = 4^k + 3\cdot 2^{k-1}+1$, e o termo de ordem zero é 1 (i.e., 1, 8, 23,
77, 281, $\ldots$). Para esta sequência de \emph{gaps} a complexidade no pior
caso é dada por $\Theta\left( n^{\sfrac{4}{3}} \right)$.

O~\autoref{lst:shellsort} mostra nossa implementação do \emph{shell sort}. Na
linha 1 temos a função de calcula os termos de ordem $k$ da sequência. na linha
3 é definida uma \emph{steam}, i.e., uma sequência sem limite definido, de
\emph{gaps} onde o primeiro termo é 1 e os demais são formados pelos termos de
ordem $k$ (com $k$ de 1 em diante) da sequência de Sedgewick. Na linha 6,
pegamos todos os elementos da \emph{stream} que sejam menores que o tamanho do
vetor em ordem reversa para formar a sequência de \emph{gaps}.

Em nossa implementação, a função \emph{shellsort} recebe dois parâmetros. O
primeiro é o vetor, \texttt{a}, que será ordenado. O segundo é uma implementação
da interface \mintinline{scala}{Ordering[T]} que permite fazer a comparação
entre dois valores de um tipo \texttt{T}. Note que o parâmetro \texttt{ord} é
definido como \mintinline{scala}{implicit}, o que significa que se houver um
valor implícito declarado no escopo da chamada da função \texttt{shellsort},
este parâmetro não precisa ser passado. Nosso programa principal define um
objeto implícito para comparação de objetos do tipo \mintinline{scala}{Person}.


\subsubsection{Algoritmo Comb Sort}
\label{sec:algoritmo-comb-sort}

O \emph{Comb sort} foi desenvolvido por Włodzimierz Dobosiewicz. Assim como o
\emph{shell sort}, o \emph{comb sort} também pode ser visto como uma
generalização de outro algoritmo de ordenação que utiliza \emph{gaps} para
definir subsequencias intercaladas, e ordenar primeiramente estas subsequencias.
No caso do \emph{comb sort}, ele é uma generalização do Algoritmo da Bolha,
notório pelo desempenho pobre. Surpreendentemente, o desempenho do algoritmo
\emph{comb sort} é consideravelmente mais rápido.

\begin{listing}
\begin{minted}[autogobble,breaklines,linenos,fontsize=\small]{scala}
  def combsort[T](a: Array[T])(implicit ord: Ordering[T]): Array[T] = {
    var gap = a.size
    val shrink = 1.3
    var sorted = false

    while (!sorted) {
      gap = floor(gap/shrink).toInt
      if (gap > 1) {
        sorted = false
      }
      else {
        gap = 1
        sorted = true
      }

      for (i <- gap until a.size) {
        if (ord.compare(a(i-gap), a(i)) > 0) {
          swap(a, i-gap, i)
          sorted = false
        }
      }
    }
    a
  }
\end{minted}
\caption{Implementação do algoritmo Comb sort.}
\label{lst:combsort}
\end{listing}

Diferentemente do que acontece com o \emph{shell sort}, o algoritmo \emph{comb
  sort} não é muito sensível às sequências de intervalos e esta sequência
tipicamente é definida como sendo $[\sfrac{n}{k}, \sfrac{n}{k^2},
\sfrac{n}{k^3}, \ldots, 1]$, onde $k$ é chamado de \emph{fator de encolhimento}.
Tipicamente, o algoritmo \emph{comb sort} apresenta seu melhor desempenho com
fator de encolhimento $k = 1.3$.


%%% ============================================================================
\section{Análise de Desempenho dos Algoritmos}
\label{sec:analise-desempenho}

Como é possível ver no~\autoref{lst:mysort-run}, a linha 6 coleta o tempo do
sistema imediatamente antes de acionar o algoritmo de ordenação selecionado, e
imediatamente após a execução do algoritmo de ordenação o tempo do sistema é
coletado novamente, na linha 13. O tempo transcorrido é relatado ao final da
execução do programa. Entretanto, há outras questões a considerar. Pode haver
variação no tempo de execução do programa em função de uma série de fatores
externos ao nosso código. Por isso, a forma mais segura de analisar a
performance é executar cada programa várias vezes para um mesmo arquivo de
entrada e trabalhar com a média dos tempos coletados. As seções seguintes
descrevem como foi feita a coleta dos dados de tempo de execução e como esses
dados foram analisados.

Todos os testes foram executado em um \emph{notebook} Acer Aspire V15, com
processador Intel Core i7, e 32Gb de RAM.


\subsection{Coleta e Pré-processamento de Dados}
\label{sec:coleta-preproc}

Como dito anteriormente, cada algoritmo de ordenação -- no nosso caso o
\emph{shell sort} e o \emph{comb sort} -- foi executado uma quantidade de vezes
sobre cada um dos arquivos de dados e os tempos de execução reportados foram
coletados e armazenados em um arquivo. Mais especificamente, cada algoritmo de
ordenação foi executado 25 vezes para cada um dos 25 arquivo de dados. O script
do interpretador de comandos Bash, apresentado na~\autoref{lst:collect-times},
realiza esta função.

Uma vez coletados os tempos de execução é necessário processar o arquivo com os
tempos de execução para extrair as médias por algoritmo e por número de
registros. Também decidimos que seria interessante extrair os tempos mínimos e
máximos de execução para cada algoritmos e arquivo de dados. Para essa
finalidade, escrevemos um \emph{script} em Scala, utilizando o shell
\href{http://ammonite.io/}{Ammonite} que é um interpretador de comandos Scala
com algumas facilidades de importação de bibliotecas e código de outros
\emph{scripts}.

O código fonte do \emph{script} criado para a preparação dos dados de tempo de
execução é mostrado na~\autoref{lst:prepare-times}. O \emph{script} espera o
identificador do algoritmo para o qual devem ser calculadas as estatísticas, o
nome do arquivo de dados de entrada (contendo os dados de tempo de execução), e
o nome do arquivo de saída, onde serão gravadas as estatísticas.

\sisetup{round-mode=places,round-precision=3}

\begin{table}
  \centering
    \csvreader[tabular=|r|r|r|r|,
    table head=\hline\textbf{n} & \textbf{Média} & \textbf{Mínimo} & \textbf{Máximo}\\\hline,
    late after line=\\\hline,
    head to column names]{../report-data/shellsort-times-2018-09-28.csv}%
    {}%
    {\num{\n} & \num{\mean} & \num{\min} & \num{\max}}
  \caption{Estatísticas dos tempos de execução do algoritmo \emph{Shell sort}.}
  \label{tab:stats-shellsort}
\end{table}

\begin{table}
  \centering
    \csvreader[tabular=|r|r|r|r|,
    table head=\hline\textbf{n} & \textbf{Média} & \textbf{Mínimo} & \textbf{Máximo}\\\hline,
    late after line=\\\hline,
    head to column names]{../report-data/combsort-times-2018-09-29.csv}%
    {}%
    {\num{\n} & \num{\mean} & \num{\min} & \num{\max}}
  \caption{Estatísticas dos tempos de execução do algoritmo \emph{Comb sort}.}
  \label{tab:stats-combsort}
\end{table}

A~\autoref{tab:stats-shellsort} e a~\autoref{tab:stats-combsort} mostram as
estatísticas extraídas dos arquivos com os tempos de execução para os algoritmos
\emph{Shell sort} e \emph{Comb sort} respectivamente. Todos os tempos são dados
em milissegundos.


\subsection{Análise dos Resultados}
\label{sec:analise-result}


\newpage

%%% ============================================================================
\appendix

\section{Scripts de Shell}
\label{sec:scripts-de-shell}

% ==============================================================================
\subsection{Script de shell para geração dos arquivos de dados}
\addcontentsline{lol}{subsection}{\thesubsection~~\,Script de shell para geração
  dos arquivos de dados.} 
\label{lst:mkdata}

\inputminted[linenos,fontsize=\footnotesize]{bash}{../mkdata.sh}


% ==============================================================================
\subsection{Script de shell para execução dos algoritmos de ordenação}
\addcontentsline{lol}{subsection}{\thesubsection~~\,Script de shell para
  execução dos algoritmos de ordenação}
\label{lst:collect-times}

\inputminted[linenos,breaklines,fontsize=\footnotesize]{bash}{../collect-times.sh}


% ==============================================================================
\subsection{Script de shell para pré-processamento do tempos de execução}
\addcontentsline{lol}{subsection}{\thesubsection~~\,Script de shell para
  pré-processamento do tempos de execução}
\label{lst:prepare-times}

\inputminted[linenos,breaklines,fontsize=\footnotesize]{scala}{../report-data/prepare.sc}






\end{document}


%%% Local Variables:
%%% mode: latex
%%% TeX-master: t
%%% ispell-local-dictionary: "brasileiro"
%%% TeX-command-extra-options: "-shell-escape"
%%% End:
